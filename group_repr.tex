\documentclass[12pt, t]{beamer}
\usepackage{amsmath}
\usepackage{bm}
\usepackage{setspace}
\usepackage[style=phys, backend=biber]{biblatex}
\usepackage[no-math]{luatexja-fontspec}
\usepackage[utf8]{inputenc}
\usepackage{bxcoloremoji}
\addbibresource{all.bib}

% 論文タイトルの大文字小文字は bib ファイルのままにする
\DeclareFieldFormat{titlecase}{#1}

\setmainjfont{SourceHanSerifJP}
\setsansjfont{SourceHanSansJP}
\ltjsetparameter{jacharrange={-2}}

\renewcommand{\kanjifamilydefault}{\gtdefault}

\setstretch{1.1}
\setlength{\parskip}{7pt}

\usetheme{CambridgeUS}
\usefonttheme{professionalfonts}

\usepackage{geometry}
\usepackage{tikz}
\usetikzlibrary{calc, decorations.pathmorphing, patterns}
\usetikzlibrary{cd}
\usepackage[nomessages]{fp}

%%%%%%% 調和振動子系
%%%% \A*cos(\OMEGA*\t)+\B*sin(1.73*\OMEGA*\t) のパラメータ
\def\A{1}
\def\B{0.8}
\def\OMEGA{0.2}

%%% 壁などの描画のパラメータ
\def\wallHeight{2}
\def\wallWidth{0.2}
\def\totalLength{10}
\def\springStraightLength{0.2}
\def\axisDepth{-0.7}
\def\t{30}

%%% 壁などのスタイル
\tikzset{wall/.style={pattern = north east lines}}
\tikzset{ball/.style={circle,shade,outer color=black!90!white,inner color=white,inner sep=2.5mm,label={$m$}}}
\tikzset{spring/.style={decorate,decoration={aspect=0.4, segment length=#1, amplitude=2mm,coil}}}
\tikzset{springk/.style={label={$k$},yshift=2}}

%%%%%%%

\newcommand{\eapple}{\coloremoji{🍎}}
\newcommand{\etangerine}{\coloremoji{🍊}}
\newcommand{\ebanana}{\coloremoji{🍌}}

\newcommand{\lr}[1]{\left({}#1\right){}}
\newcommand{\slr}[1]{\left[{}#1\right]{}}
\newcommand{\clr}[1]{\left\{{}#1\right\}{}}

\newcommand{\eAEB}{\slr{\eapple{}\etangerine{}\ebanana{}}}
\newcommand{\eABE}{\slr{\eapple{}\ebanana{}\etangerine{}}}
\newcommand{\eEAB}{\slr{\etangerine{}\eapple{}\ebanana{}}}
\newcommand{\eEBA}{\slr{\etangerine{}\ebanana{}\eapple{}}}
\newcommand{\eBAE}{\slr{\ebanana{}\eapple{}\etangerine{}}}
\newcommand{\eBEA}{\slr{\ebanana{}\etangerine{}\eapple{}}}

\def\opcty{0.05}

\title{群の表現}
\author{武智}
\begin{document}
\frame{\maketitle}

\begin{frame}
\frametitle{はじめに}
現代数学では、ほとんど至るところで群論が使われている。

物理学でも使われているが、群論の言葉を明示的には使わずに議論を進めることも多い(当社比)。

情報科学では、暗号理論など一部を除いて、あからさまに群論が使われることは少ない(個人の感想です)。
\end{frame}

\begin{frame}
\frametitle{はじめに}
Q. 場の理論、楕円暗号のような、高度に抽象化された場面でしか群論は使えないのだろうか?

$\rightsquigarrow$ A. そんなことはない。

群は色々なところに潜んでいる。ただ、シンプルな系では、群論を意識せずに問題が解けてしまうことが多いため、
その御利益に気付きにくい。
\end{frame}

\begin{frame}
\frametitle{はじめに}
群は代数的な定義されるので、そのままでは実体が捉えにくい。

そこで、群を線形代数と結び付けて、そこでの振る舞いから群を理解しよう、
というのが「群の表現」という研究分野。

線形代数という舞台の上では、具体的な計算もできるし、幾何学的なイメージも持てる。

群の表現を知ることで、より群が身近なものになる(かも)。
\end{frame}

\begin{frame}{流れ}
\begin{enumerate}
\item 群とは
\begin{enumerate}
\item 置換群
\item 群の作用
\end{enumerate}
\item 群の表現とは
\end{enumerate}
\end{frame}

\begin{frame}
\frametitle{群とは}
初めて群を定義したのはGalois(1832)。

群の定義には至っていないものの、先駆的な研究としてLagrange(1771)やAbel(1824), Ruffini(1799)がある。

\end{frame}

\begin{frame}
\frametitle{置換群}
Galoisが導入したのは今で言う置換群。

元々Galoisは「置換の集まり」の意味で``groupe de substitusions''と呼んでいた。
繰り返し言及するうちに、一般名詞 groupe が数学用語になっていったらしい。

Galoisが数学用語としての groupe の定義を書き下したのは、決闘前夜だったという\cite{Neumann2011}。
\end{frame}

\begin{frame}
\frametitle{置換}
\eapple{}と\etangerine{}と\ebanana{}の並び方は
\begin{align}
  \Omega = \clr{
  \slr{\eapple{}\etangerine{}\ebanana{}},
  \slr{\eapple{}\ebanana{}\etangerine{}},
  \slr{\etangerine{}\ebanana{}\eapple{}},
  \slr{\etangerine{}\eapple{}\ebanana{}},
  \slr{\ebanana{}\eapple{}\etangerine{}},
  \slr{\ebanana{}\etangerine{}\eapple{}}} \notag
\end{align}
の$6$通り。
\end{frame}

\begin{frame}[fragile]
\frametitle{並び方の間の変換}
並び方の間の変換を全て考える。$6^2 = 36$個。
\[
\begin{tikzcd}
&
\eAEB
 \arrow[loop, out=120, in=60, looseness=3]
 \arrow[r, leftrightarrow]
 \arrow[rrd, leftrightarrow]
 \arrow[rdd, leftrightarrow]
 \arrow[dd, leftrightarrow]
 \arrow[ld, leftrightarrow]
&
\eABE
 \arrow[loop, out=120, in=60, looseness=3]
 \arrow[rd, leftrightarrow]
 \arrow[dd, leftrightarrow]
 \arrow[ldd, leftrightarrow]
 \arrow[lld, leftrightarrow]
&
\\
\eEAB
 \arrow[loop, out=120, in=60, looseness=3]
 \arrow[rrr, leftrightarrow]
 \arrow[rrd, leftrightarrow]
 \arrow[rd, leftrightarrow]
&
&
&
\eBAE
 \arrow[loop, out=120, in=60, looseness=3]
 \arrow[ld, leftrightarrow]
 \arrow[lld, leftrightarrow]
\\
&
\eEBA
 \arrow[loop, out=240, in=300, looseness=3]
 \arrow[r, leftrightarrow]
&
\eBEA
 \arrow[loop, out=240, in=300, looseness=3]
& 
\end{tikzcd}
\]
$6$個の並び方それぞれから$6$本の矢印が生える。
\end{frame}

\begin{frame}[fragile]
\frametitle{$e$}
自分自身へ戻る変換(恒等変換)$6$本をまとめて$e$と書く。
\[
\begin{tikzcd}
&
\eAEB
 \arrow[loop, out=120, in=60, looseness=3]
 \arrow[r, leftrightarrow, opacity=\opcty]
 \arrow[rrd, leftrightarrow, opacity=\opcty]
 \arrow[rdd, leftrightarrow, opacity=\opcty]
 \arrow[dd, leftrightarrow, opacity=\opcty]
 \arrow[ld, leftrightarrow, opacity=\opcty]
&
\eABE
 \arrow[loop, out=120, in=60, looseness=3]
 \arrow[rd, leftrightarrow, opacity=\opcty]
 \arrow[dd, leftrightarrow, opacity=\opcty]
 \arrow[ldd, leftrightarrow, opacity=\opcty]
 \arrow[lld, leftrightarrow, opacity=\opcty]
&
\\
\eEAB
 \arrow[loop, out=120, in=60, looseness=3]
 \arrow[rrr, leftrightarrow, opacity=\opcty]
 \arrow[rrd, leftrightarrow, opacity=\opcty]
 \arrow[rd, leftrightarrow, opacity=\opcty]
&
&
&
\eBAE
 \arrow[loop, out=120, in=60, looseness=3]
 \arrow[ld, leftrightarrow, opacity=\opcty]
 \arrow[lld, leftrightarrow, opacity=\opcty]
\\
&
\eEBA
 \arrow[loop, out=240, in=300, looseness=3]
 \arrow[r, leftrightarrow, opacity=\opcty]
&
\eBEA
 \arrow[loop, out=240, in=300, looseness=3]
& 
\end{tikzcd}
\]
$e$は$\Omega \rightarrow \Omega$の全単射となる。
\end{frame}


\begin{frame}[fragile]
\frametitle{$\sigma_{12}$}
$1$番目と$2$番目を入れ替える変換 $\sigma_{12}$
\[
\begin{tikzcd}
&
\eAEB
 \arrow[loop, out=120, in=60, looseness=3, opacity=\opcty]
 \arrow[r, leftrightarrow, opacity=\opcty]
 \arrow[rrd, leftrightarrow, opacity=\opcty]
 \arrow[rdd, leftrightarrow, opacity=\opcty]
 \arrow[dd, leftrightarrow, opacity=\opcty]
 \arrow[ld, leftrightarrow]
&
\eABE
 \arrow[loop, out=120, in=60, looseness=3, opacity=\opcty]
 \arrow[rd, leftrightarrow]
 \arrow[dd, leftrightarrow, opacity=\opcty]
 \arrow[ldd, leftrightarrow, opacity=\opcty]
 \arrow[lld, leftrightarrow, opacity=\opcty]
&
\\
\eEAB
 \arrow[loop, out=120, in=60, looseness=3, opacity=\opcty]
 \arrow[rrr, leftrightarrow, opacity=\opcty]
 \arrow[rrd, leftrightarrow, opacity=\opcty]
 \arrow[rd, leftrightarrow, opacity=\opcty]
&
&
&
\eBAE
 \arrow[loop, out=120, in=60, looseness=3, opacity=\opcty]
 \arrow[ld, leftrightarrow, opacity=\opcty]
 \arrow[lld, leftrightarrow, opacity=\opcty]
\\
&
\eEBA
 \arrow[loop, out=240, in=300, looseness=3, opacity=\opcty]
 \arrow[r, leftrightarrow]
&
\eBEA
 \arrow[loop, out=240, in=300, looseness=3, opacity=\opcty]
& 
\end{tikzcd}
\]
\end{frame}

\begin{frame}[fragile]
\frametitle{$\sigma_{23}$}
$2$番目と$3$番目を入れ替える変換 $\sigma_{23}$
\[
\begin{tikzcd}
&
\eAEB
 \arrow[loop, out=120, in=60, looseness=3, opacity=\opcty]
 \arrow[r, leftrightarrow]
 \arrow[rrd, leftrightarrow, opacity=\opcty]
 \arrow[rdd, leftrightarrow, opacity=\opcty]
 \arrow[dd, leftrightarrow, opacity=\opcty]
 \arrow[ld, leftrightarrow, opacity=\opcty]
&
\eABE
 \arrow[loop, out=120, in=60, looseness=3, opacity=\opcty]
 \arrow[rd, leftrightarrow, opacity=\opcty]
 \arrow[dd, leftrightarrow, opacity=\opcty]
 \arrow[ldd, leftrightarrow, opacity=\opcty]
 \arrow[lld, leftrightarrow, opacity=\opcty]
&
\\
\eEAB
 \arrow[loop, out=120, in=60, looseness=3, opacity=\opcty]
 \arrow[rrr, leftrightarrow, opacity=\opcty]
 \arrow[rrd, leftrightarrow, opacity=\opcty]
 \arrow[rd, leftrightarrow]
&
&
&
\eBAE
 \arrow[loop, out=120, in=60, looseness=3, opacity=\opcty]
 \arrow[ld, leftrightarrow]
 \arrow[lld, leftrightarrow, opacity=\opcty]
\\
&
\eEBA
 \arrow[loop, out=240, in=300, looseness=3, opacity=\opcty]
 \arrow[r, leftrightarrow, opacity=\opcty]
&
\eBEA
 \arrow[loop, out=240, in=300, looseness=3, opacity=\opcty]
& 
\end{tikzcd}
\]
\end{frame}

\begin{frame}[fragile]
\frametitle{$\sigma_{31}$}
$3$番目と$1$番目を入れ替える変換 $\sigma_{31}$
\[
\begin{tikzcd}
&
\eAEB
 \arrow[loop, out=120, in=60, looseness=3, opacity=\opcty]
 \arrow[r, leftrightarrow, opacity=\opcty]
 \arrow[rrd, leftrightarrow, opacity=\opcty]
 \arrow[rdd, leftrightarrow]
 \arrow[dd, leftrightarrow, opacity=\opcty]
 \arrow[ld, leftrightarrow, opacity=\opcty]
&
\eABE
 \arrow[loop, out=120, in=60, looseness=3, opacity=\opcty]
 \arrow[rd, leftrightarrow, opacity=\opcty]
 \arrow[dd, leftrightarrow, opacity=\opcty]
 \arrow[ldd, leftrightarrow]
 \arrow[lld, leftrightarrow, opacity=\opcty]
&
\\
\eEAB
 \arrow[loop, out=120, in=60, looseness=3, opacity=\opcty]
 \arrow[rrr, leftrightarrow]
 \arrow[rrd, leftrightarrow, opacity=\opcty]
 \arrow[rd, leftrightarrow, opacity=\opcty]
&
&
&
\eBAE
 \arrow[loop, out=120, in=60, looseness=3, opacity=\opcty]
 \arrow[ld, leftrightarrow, opacity=\opcty]
 \arrow[lld, leftrightarrow, opacity=\opcty]
\\
&
\eEBA
 \arrow[loop, out=240, in=300, looseness=3, opacity=\opcty]
 \arrow[r, leftrightarrow, opacity=\opcty]
&
\eBEA
 \arrow[loop, out=240, in=300, looseness=3, opacity=\opcty]
& 
\end{tikzcd}
\]
\end{frame}

\begin{frame}[fragile]
\frametitle{$\sigma_{123}$}
$1$番目$\rightarrow$$2$番目$\rightarrow$$3$番目$\rightarrow$$1$番目と入れ替える変換 $\sigma_{123}$
\[
\begin{tikzcd}
&
\eAEB
 \arrow[loop, out=120, in=60, looseness=3, opacity=\opcty]
 \arrow[r, leftrightarrow, opacity=\opcty]
 \arrow[rrd, rightarrow]
 \arrow[rrd, leftrightarrow, opacity=\opcty]
 \arrow[rdd, leftrightarrow, opacity=\opcty]
 \arrow[dd, leftrightarrow, opacity=\opcty]
 \arrow[ld, leftrightarrow, opacity=\opcty]
&
\eABE
 \arrow[loop, out=120, in=60, looseness=3, opacity=\opcty]
 \arrow[rd, leftrightarrow, opacity=\opcty]
 \arrow[dd, leftrightarrow, opacity=\opcty]
 \arrow[ldd, leftrightarrow, opacity=\opcty]
 \arrow[lld, rightarrow]
 \arrow[lld, leftrightarrow, opacity=\opcty]
&
\\
\eEAB
 \arrow[loop, out=120, in=60, looseness=3, opacity=\opcty]
 \arrow[rrr, leftrightarrow, opacity=\opcty]
 \arrow[rrd, rightarrow]
 \arrow[rrd, leftrightarrow, opacity=\opcty]
 \arrow[rd, leftrightarrow, opacity=\opcty]
&
&
&
\eBAE
 \arrow[loop, out=120, in=60, looseness=3, opacity=\opcty]
 \arrow[ld, leftrightarrow, opacity=\opcty]
 \arrow[lld, rightarrow]
 \arrow[lld, leftrightarrow, opacity=\opcty]
\\
&
\eEBA
 \arrow[loop, out=240, in=300, looseness=3, opacity=\opcty]
 \arrow[r, leftrightarrow, opacity=\opcty]
 \arrow[uu, rightarrow]
&
\eBEA
 \arrow[loop, out=240, in=300, looseness=3, opacity=\opcty]
 \arrow[uu, rightarrow]
& 
\end{tikzcd}
\]
\end{frame}

\begin{frame}[fragile]
\frametitle{$\sigma_{321}$}
$3$番目$\rightarrow$$2$番目$\rightarrow$$1$番目$\rightarrow$$3$番目と入れ替える変換 $\sigma_{321}$
\[
\begin{tikzcd}
&
\eAEB
 \arrow[loop, out=120, in=60, looseness=3, opacity=\opcty]
 \arrow[r, leftrightarrow, opacity=\opcty]
 \arrow[rrd, leftarrow]
 \arrow[rrd, leftrightarrow, opacity=\opcty]
 \arrow[rdd, leftrightarrow, opacity=\opcty]
 \arrow[dd, leftrightarrow, opacity=\opcty]
 \arrow[ld, leftrightarrow, opacity=\opcty]
&
\eABE
 \arrow[loop, out=120, in=60, looseness=3, opacity=\opcty]
 \arrow[rd, leftrightarrow, opacity=\opcty]
 \arrow[dd, leftrightarrow, opacity=\opcty]
 \arrow[ldd, leftrightarrow, opacity=\opcty]
 \arrow[lld, leftarrow]
 \arrow[lld, leftrightarrow, opacity=\opcty]
&
\\
\eEAB
 \arrow[loop, out=120, in=60, looseness=3, opacity=\opcty]
 \arrow[rrr, leftrightarrow, opacity=\opcty]
 \arrow[rrd, leftarrow]
 \arrow[rrd, leftrightarrow, opacity=\opcty]
 \arrow[rd, leftrightarrow, opacity=\opcty]
&
&
&
\eBAE
 \arrow[loop, out=120, in=60, looseness=3, opacity=\opcty]
 \arrow[ld, leftrightarrow, opacity=\opcty]
 \arrow[lld, leftarrow]
 \arrow[lld, leftrightarrow, opacity=\opcty]
\\
&
\eEBA
 \arrow[loop, out=240, in=300, looseness=3, opacity=\opcty]
 \arrow[r, leftrightarrow, opacity=\opcty]
 \arrow[uu, leftarrow]
&
\eBEA
 \arrow[loop, out=240, in=300, looseness=3, opacity=\opcty]
 \arrow[uu, leftarrow]
& 
\end{tikzcd}
\]
\end{frame}

\begin{frame}
\frametitle{変換の間の関係}
この$6$種類の変換をまとめて$S_3$と書く。
\begin{align}
  S_3 = \clr{e, \sigma_{12}, \sigma_{23}, \sigma_{31}, \sigma_{123}, \sigma_{321}}
\end{align}
$S_3$の元は$\Omega \rightarrow \Omega$の写像なので、合成写像を考えることができる。

例えば
$e(\sigma_{12}(\Omega)) = \sigma_{12}(\Omega)$なので$e$と$\sigma_{12}$の合成写像は$\sigma_{12}$.
\end{frame}

\begin{frame}[fragile]
\frametitle{$\sigma_{12} \cdot \sigma_{123} = \sigma_{23}$}
${\color{blue} \sigma_{12}}(\sigma_{123}(\Omega)) = {\color{red} \sigma_{23}}(\Omega)$
\[
\begin{tikzcd}
&
\eAEB
 \arrow[loop, out=120, in=60, looseness=3, opacity=\opcty]
 \arrow[r, leftrightarrow, red]
 \arrow[r, leftrightarrow, opacity=\opcty]
 \arrow[rrd, rightarrow]
 \arrow[rrd, leftrightarrow, opacity=\opcty]
 \arrow[rdd, leftrightarrow, opacity=\opcty]
 \arrow[dd, leftrightarrow, opacity=\opcty]
 \arrow[ld, leftrightarrow, blue]
&
\eABE
 \arrow[loop, out=120, in=60, looseness=3, opacity=\opcty]
 \arrow[rd, leftrightarrow, blue]
 \arrow[dd, leftrightarrow, opacity=\opcty]
 \arrow[ldd, leftrightarrow, opacity=\opcty]
 \arrow[lld, rightarrow]
 \arrow[lld, leftrightarrow, opacity=\opcty]
&
\\
\eEAB
 \arrow[loop, out=120, in=60, looseness=3, opacity=\opcty]
 \arrow[rrr, leftrightarrow, opacity=\opcty]
 \arrow[rrd, rightarrow]
 \arrow[rrd, leftrightarrow, opacity=\opcty]
 \arrow[rd, leftrightarrow, red]
&
&
&
\eBAE
 \arrow[loop, out=120, in=60, looseness=3, opacity=\opcty]
 \arrow[ld, leftrightarrow, red]
 \arrow[lld, rightarrow]
 \arrow[lld, leftrightarrow, opacity=\opcty]
\\
&
\eEBA
 \arrow[loop, out=240, in=300, looseness=3, opacity=\opcty]
 \arrow[r, leftrightarrow, blue]
 \arrow[uu, rightarrow]
&
\eBEA
 \arrow[loop, out=240, in=300, looseness=3, opacity=\opcty]
 \arrow[uu, rightarrow]
& 
\end{tikzcd}
\]
\end{frame}

\begin{frame}[fragile]
\frametitle{$\sigma_{31} \cdot \sigma_{12} = \sigma_{123}$}
${\color{blue} \sigma_{31}}(\sigma_{12}(\Omega)) = {\color{red} \sigma_{123}}(\Omega)$
\[
\begin{tikzcd}
&
\eAEB
 \arrow[loop, out=120, in=60, looseness=3, opacity=\opcty]
 \arrow[r, leftrightarrow, opacity=\opcty]
 \arrow[r, leftrightarrow, opacity=\opcty]
 \arrow[rrd, rightarrow, red]
 \arrow[rrd, leftrightarrow, opacity=\opcty]
 \arrow[rdd, leftrightarrow, blue]
 \arrow[dd, leftrightarrow, opacity=\opcty]
 \arrow[ld, leftrightarrow]
&
\eABE
 \arrow[loop, out=120, in=60, looseness=3, opacity=\opcty]
 \arrow[rd, leftrightarrow]
 \arrow[dd, leftrightarrow, opacity=\opcty]
 \arrow[ldd, leftrightarrow, blue]
 \arrow[lld, rightarrow, red]
 \arrow[lld, leftrightarrow, opacity=\opcty]
&
\\
\eEAB
 \arrow[loop, out=120, in=60, looseness=3, opacity=\opcty]
 \arrow[rrr, leftrightarrow, blue]
 \arrow[rrd, rightarrow, red]
 \arrow[rrd, leftrightarrow, opacity=\opcty]
 \arrow[rd, leftrightarrow, opacity=\opcty]
&
&
&
\eBAE
 \arrow[loop, out=120, in=60, looseness=3, opacity=\opcty]
 \arrow[ld, leftrightarrow, opacity=\opcty]
 \arrow[lld, rightarrow, red]
 \arrow[lld, leftrightarrow, opacity=\opcty]
\\
&
\eEBA
 \arrow[loop, out=240, in=300, looseness=3, opacity=\opcty]
 \arrow[r, leftrightarrow]
 \arrow[uu, rightarrow, red]
&
\eBEA
 \arrow[loop, out=240, in=300, looseness=3, opacity=\opcty]
 \arrow[uu, rightarrow, red]
& 
\end{tikzcd}
\]
\end{frame}

\begin{frame}
\frametitle{互換と、置換の偶奇}
$\sigma_{123}(\Omega) = \sigma_{12}(\sigma_{23}(\Omega)) = \sigma_{31}(\sigma_{12}(\Omega))$.

$\sigma_{12}$のような「$2$個の交換」を特に\alert{互換}と呼ぶ。
どんな並び方の変換(並び替え)も互換の組み合わせで書ける(cf. あみだくじ)。

並び替えを互換で表す組み合わせは色々あり得るが、互換個数の偶奇は変わらない。
\vspace{-1\zw}
\begin{center}
{\tiny
\begin{tabular}{cccccc}
&&&&& \\
\multicolumn{2}{c}{\eapple} & \multicolumn{2}{c}{\etangerine} & \multicolumn{2}{c}{\ebanana} \\
& \multicolumn{2}{|c|}{} & \multicolumn{2}{|c|}{} & \\
& \multicolumn{2}{|c|}{} & \multicolumn{2}{|c|}{} & \\
& \multicolumn{2}{|c|}{} & \multicolumn{2}{|c|}{} & \\\cline{4-5}
& \multicolumn{2}{|c|}{} & \multicolumn{2}{|c|}{} & \\\cline{2-3}
& \multicolumn{2}{|c|}{} & \multicolumn{2}{|c|}{} & \\
\multicolumn{2}{c}{\ebanana} & \multicolumn{2}{c}{\eapple} & \multicolumn{2}{c}{\etangerine}
\end{tabular}
$=$
\begin{tabular}{cccccc}
&&&&& \\
\multicolumn{2}{c}{\eapple} & \multicolumn{2}{c}{\etangerine} & \multicolumn{2}{c}{\ebanana} \\
& \multicolumn{2}{|c|}{} & \multicolumn{2}{|c|}{} & \\\cline{2-3}
& \multicolumn{2}{|c|}{} & \multicolumn{2}{|c|}{} & \\\cline{2-3}
& \multicolumn{2}{|c|}{} & \multicolumn{2}{|c|}{} & \\\cline{4-5}
& \multicolumn{2}{|c|}{} & \multicolumn{2}{|c|}{} & \\\cline{2-3}
& \multicolumn{2}{|c|}{} & \multicolumn{2}{|c|}{} & \\
\multicolumn{2}{c}{\ebanana} & \multicolumn{2}{c}{\eapple} & \multicolumn{2}{c}{\etangerine}
\end{tabular}
}
\end{center}

互換の個数が偶数(奇数)になる置換を偶置換(奇置換)と呼ぶ。
\end{frame}


\begin{frame}
\frametitle{合成$=$積}
写像の合成を、積の演算とみなす。$\sigma(\sigma'(\Omega)) = (\sigma \cdot \sigma')(\Omega)$
\begin{center}
\begin{tabular}{c|cccccc}
  $\sigma \backslash \sigma'$ & $e$            & $\sigma_{12}$  & $\sigma_{23}$  & $\sigma_{31}$  & $\sigma_{123}$ & $\sigma_{321}$ \\ \hline
  $e$                         & $e$            & $\sigma_{12}$  & $\sigma_{23}$  & $\sigma_{31}$  & $\sigma_{123}$ & $\sigma_{321}$ \\
  $\sigma_{12}$               & $\sigma_{12}$  & $e$            & $\sigma_{123}$ & $\sigma_{321}$ & $\sigma_{23}$  & $\sigma_{31}$  \\
  $\sigma_{23}$               & $\sigma_{23}$  & $\sigma_{321}$ & $e$            & $\sigma_{123}$ & $\sigma_{31}$  & $\sigma_{12}$  \\
  $\sigma_{31}$               & $\sigma_{31}$  & $\sigma_{123}$ & $\sigma_{321}$ & $e$            & $\sigma_{12}$  & $\sigma_{23}$  \\
  $\sigma_{123}$              & $\sigma_{123}$ & $\sigma_{31}$  & $\sigma_{12}$  & $\sigma_{23}$  & $\sigma_{321}$ & $e$            \\
  $\sigma_{321}$              & $\sigma_{321}$ & $\sigma_{23}$  & $\sigma_{31}$  & $\sigma_{12}$  & $e$            & $\sigma_{123}$
\end{tabular}
\end{center}
この表だけあれば、並び方$\Omega$の写像であることを忘れて
$S_3$だけで計算ができる。$\rightsquigarrow$ $S_3$を改めて数学的対象と考える。$\rightsquigarrow$ 群
\end{frame}


\begin{frame}
\frametitle{群の定義}
空でない集合$G$に$2$項演算$\cdot$が定義されていて、次の$3$条件
\begin{enumerate}
\item 結合則:任意の$a, b, c \in G$について $(a \cdot b) \cdot c = a \cdot (b \cdot c)$となる。
\item 単位元の存在:任意の$g \in G$について $g \cdot e = e \cdot g = g$ となる元 $e \in G$が存在する。
\item 逆元の存在:任意の$g \in G$について$g \cdot g^{-1} = g^{-1} \cdot g = e$ となる元 $g^{-1} \in G$が$g$ごとに存在する。
\end{enumerate}
を満たすとき、$G$は群であると言う。

$S_3$が群であることを確認しよう。
\end{frame}

\begin{frame}[fragile]
\frametitle{結合則}
結合則は、$\sigma \in S_3$が写像なので自然に成立。
\[
\begin{tikzcd}
&
\eAEB
 \arrow[loop, out=120, in=60, looseness=3, opacity=\opcty]
 \arrow[r, leftrightarrow, opacity=\opcty]
 \arrow[rrd, leftrightarrow, opacity=\opcty]
 \arrow[rdd, leftrightarrow, opacity=\opcty]
 \arrow[dd, leftrightarrow, opacity=\opcty]
 \arrow[ld, leftrightarrow, opacity=\opcty]
 \arrow[ld, rightarrow]
&
\eABE
 \arrow[loop, out=120, in=60, looseness=3, opacity=\opcty]
 \arrow[rd, leftrightarrow, opacity=\opcty]
 \arrow[dd, leftrightarrow, opacity=\opcty]
 \arrow[ldd, leftrightarrow, opacity=\opcty]
 \arrow[lld, leftrightarrow, opacity=\opcty]
&
\\
\eEAB
 \arrow[loop, out=120, in=60, looseness=3, opacity=\opcty]
 \arrow[rrr, leftrightarrow, opacity=\opcty]
 \arrow[rrd, leftrightarrow, opacity=\opcty]
 \arrow[rd, leftrightarrow, opacity=\opcty]
 \arrow[rd, rightarrow, blue]
&
&
&
\eBAE
 \arrow[loop, out=120, in=60, looseness=3, opacity=\opcty]
 \arrow[ld, leftrightarrow, opacity=\opcty]
 \arrow[lld, leftrightarrow, opacity=\opcty]
\\
&
\eEBA
 \arrow[loop, out=240, in=300, looseness=3, opacity=\opcty]
 \arrow[r, leftrightarrow, opacity=\opcty]
 \arrow[ruu, rightarrow, red]
&
\eBEA
 \arrow[loop, out=240, in=300, looseness=3, opacity=\opcty]
& 
\end{tikzcd}
\]
${\color{red} \sigma_{31}} \cdot {\color{blue} \sigma_{23}} \cdot \sigma_{12}$
\end{frame}

\begin{frame}[fragile]
\frametitle{結合則}
結合則は、$\sigma \in S_3$が写像なので自然に成立。
\[
\begin{tikzcd}
&
\eAEB
 \arrow[loop, out=120, in=60, looseness=3, opacity=\opcty]
 \arrow[r, leftrightarrow, opacity=\opcty]
 \arrow[rrd, leftrightarrow, opacity=\opcty]
 \arrow[rdd, leftrightarrow, opacity=\opcty]
 \arrow[dd, leftrightarrow, opacity=\opcty]
 \arrow[ld, leftrightarrow, opacity=\opcty]
 \arrow[ld, rightarrow, opacity=0.1]
 \arrow[dd, rightarrow, green]
&
\eABE
 \arrow[loop, out=120, in=60, looseness=3, opacity=\opcty]
 \arrow[rd, leftrightarrow, opacity=\opcty]
 \arrow[dd, leftrightarrow, opacity=\opcty]
 \arrow[ldd, leftrightarrow, opacity=\opcty]
 \arrow[lld, leftrightarrow, opacity=\opcty]
&
\\
\eEAB
 \arrow[loop, out=120, in=60, looseness=3, opacity=\opcty]
 \arrow[rrr, leftrightarrow, opacity=\opcty]
 \arrow[rrd, leftrightarrow, opacity=\opcty]
 \arrow[rd, leftrightarrow, opacity=\opcty]
 \arrow[rd, rightarrow, blue, opacity=0.1]
&
&
&
\eBAE
 \arrow[loop, out=120, in=60, looseness=3, opacity=\opcty]
 \arrow[ld, leftrightarrow, opacity=\opcty]
 \arrow[lld, leftrightarrow, opacity=\opcty]
\\
&
\eEBA
 \arrow[loop, out=240, in=300, looseness=3, opacity=\opcty]
 \arrow[r, leftrightarrow, opacity=\opcty]
 \arrow[ruu, rightarrow, red]
&
\eBEA
 \arrow[loop, out=240, in=300, looseness=3, opacity=\opcty]
& 
\end{tikzcd}
\]
${\color{red} \sigma_{31}} \cdot {\color{blue} \sigma_{23}} \cdot \sigma_{12}
= {\color{red} \sigma_{31}} \cdot ({\color{blue} \sigma_{23}} \cdot \sigma_{12})
= {\color{red} \sigma_{31}} \cdot {\color{green} \sigma_{321}}$
\end{frame}

\begin{frame}[fragile]
\frametitle{結合則}
結合則は、$\sigma \in S_3$が写像なので自然に成立。
\[
\begin{tikzcd}
&
\eAEB
 \arrow[loop, out=120, in=60, looseness=3, opacity=\opcty]
 \arrow[r, leftrightarrow, opacity=\opcty]
 \arrow[rrd, leftrightarrow, opacity=\opcty]
 \arrow[rdd, leftrightarrow, opacity=\opcty]
 \arrow[dd, leftrightarrow, opacity=\opcty]
 \arrow[ld, leftrightarrow, opacity=\opcty]
 \arrow[ld, rightarrow]
&
\eABE
 \arrow[loop, out=120, in=60, looseness=3, opacity=\opcty]
 \arrow[rd, leftrightarrow, opacity=\opcty]
 \arrow[dd, leftrightarrow, opacity=\opcty]
 \arrow[ldd, leftrightarrow, opacity=\opcty]
 \arrow[lld, leftrightarrow, opacity=\opcty]
&
\\
\eEAB
 \arrow[loop, out=120, in=60, looseness=3, opacity=\opcty]
 \arrow[rrr, leftrightarrow, opacity=\opcty]
 \arrow[rrd, leftrightarrow, opacity=\opcty]
 \arrow[rd, leftrightarrow, opacity=\opcty]
 \arrow[rd, rightarrow, blue, opacity=0.1]
 \arrow[rru, rightarrow, purple]
&
&
&
\eBAE
 \arrow[loop, out=120, in=60, looseness=3, opacity=\opcty]
 \arrow[ld, leftrightarrow, opacity=\opcty]
 \arrow[lld, leftrightarrow, opacity=\opcty]
\\
&
\eEBA
 \arrow[loop, out=240, in=300, looseness=3, opacity=\opcty]
 \arrow[r, leftrightarrow, opacity=\opcty]
 \arrow[ruu, rightarrow, red, opacity=0.1]
&
\eBEA
 \arrow[loop, out=240, in=300, looseness=3, opacity=\opcty]
& 
\end{tikzcd}
\]
${\color{red} \sigma_{31}} \cdot {\color{blue} \sigma_{23}} \cdot \sigma_{12}
= ({\color{red} \sigma_{31}} \cdot {\color{blue} \sigma_{23}}) \cdot \sigma_{12}
= {\color{purple} \sigma_{321}} \cdot \sigma_{12}$
\end{frame}


\begin{frame}[fragile]
\frametitle{単位元}
単位元は恒等写像 $e$.
\[
\begin{tikzcd}
&
\eAEB
 \arrow[loop, out=120, in=60, looseness=3]
 \arrow[r, leftrightarrow, opacity=\opcty]
 \arrow[rrd, leftrightarrow, opacity=\opcty]
 \arrow[rdd, leftrightarrow, opacity=\opcty]
 \arrow[dd, leftrightarrow, opacity=\opcty]
 \arrow[ld, leftrightarrow, opacity=\opcty]
&
\eABE
 \arrow[loop, out=120, in=60, looseness=3]
 \arrow[rd, leftrightarrow, opacity=\opcty]
 \arrow[dd, leftrightarrow, opacity=\opcty]
 \arrow[ldd, leftrightarrow, opacity=\opcty]
 \arrow[lld, leftrightarrow, opacity=\opcty]
&
\\
\eEAB
 \arrow[loop, out=120, in=60, looseness=3]
 \arrow[rrr, leftrightarrow, opacity=\opcty]
 \arrow[rrd, leftrightarrow, opacity=\opcty]
 \arrow[rd, leftrightarrow, opacity=\opcty]
&
&
&
\eBAE
 \arrow[loop, out=120, in=60, looseness=3]
 \arrow[ld, leftrightarrow, opacity=\opcty]
 \arrow[lld, leftrightarrow, opacity=\opcty]
\\
&
\eEBA
 \arrow[loop, out=240, in=300, looseness=3]
 \arrow[r, leftrightarrow, opacity=\opcty]
&
\eBEA
 \arrow[loop, out=240, in=300, looseness=3]
& 
\end{tikzcd}
\]
\end{frame}

\begin{frame}[fragile]
\frametitle{逆元}
逆元も、$\sigma \in S_3$が写像(矢印)なので自然に存在。
\[
\begin{tikzcd}
&
\eAEB
 \arrow[loop, out=120, in=60, looseness=3]
 \arrow[r, leftrightarrow]
 \arrow[rrd, leftrightarrow]
 \arrow[rdd, leftrightarrow]
 \arrow[dd, leftrightarrow]
 \arrow[ld, leftrightarrow]
&
\eABE
 \arrow[loop, out=120, in=60, looseness=3]
 \arrow[rd, leftrightarrow]
 \arrow[dd, leftrightarrow]
 \arrow[ldd, leftrightarrow]
 \arrow[lld, leftrightarrow]
&
\\
\eEAB
 \arrow[loop, out=120, in=60, looseness=3]
 \arrow[rrr, leftrightarrow]
 \arrow[rrd, leftrightarrow]
 \arrow[rd, leftrightarrow]
&
&
&
\eBAE
 \arrow[loop, out=120, in=60, looseness=3]
 \arrow[ld, leftrightarrow]
 \arrow[lld, leftrightarrow]
\\
&
\eEBA
 \arrow[loop, out=240, in=300, looseness=3]
 \arrow[r, leftrightarrow]
&
\eBEA
 \arrow[loop, out=240, in=300, looseness=3]
& 
\end{tikzcd}
\]
$\therefore$ $S_3$は群。$S_3$は$3$次対称群と呼ばれる。
\end{frame}

\begin{frame}
\frametitle{自明な群}
最も簡単な群は、恒等変換$e$だけから成る群である。
\begin{enumerate}
\item 結合側:$(e \cdot e) \cdot e = e \cdot (e \cdot e)$
\item 単位元:$e \cdot e = e$
\item 逆元:$e \cdot e = e \Rightarrow e^{-1} = e$
\end{enumerate}
この$\clr{e}$を自明な群と呼ぶ。
\end{frame}

\begin{frame}
\frametitle{部分群}
ある群の部分集合だけで群になることがある。

例えば$S_3$の中で
\begin{align}
  \Sigma_{12} &= \clr{e,\ \sigma_{12}},\
  \Sigma_{23} = \clr{e,\ \sigma_{23}},\
  \Sigma_{31} = \clr{e,\ \Sigma_{31}},\\
  A_3 &= \clr{e,\ \sigma_{123},\ \sigma_{321}}
\end{align}
はそれぞれ群になる。\\このようなとき、$A_3$は$S_3$の部分群であると言う。
\end{frame}

\begin{frame}
\frametitle{$3$次交代群}
$A_3$は$S_3$の中の偶置換を集めたもの(偶置換同士の積は偶置換)。
\begin{tabular}{c|ccc}
                 &$e$           &$\sigma_{123}$&$\sigma_{321}$\\ \hline
  $e$            &$e$           &$\sigma_{123}$&$\sigma_{321}$\\
  $\sigma_{123}$ &$\sigma_{123}$&$\sigma_{321}$&$e$ \\
  $\sigma_{321}$ &$\sigma_{321}$&$e$           &$\sigma_{123}$
\end{tabular}

なお、構成要素が$3$個の群は必ず$A_3$と同じ(同型)になる。
\end{frame}

\begin{frame}
\frametitle{$3$次交代群}
互いに異なる元$3$個からなる群を$G_3$とする。
$G_3$の元のうち$1$つは単位元$e$なので$G_3 = \clr{e, g, h}$と書いておく。

積の表は、単位元の性質から
\begin{center}
\begin{tabular}{c|ccc}
      &$e$&$g$&$h$\\ \hline
  $e$ &$e$&$g$&$h$\\
  $g$ &$g$&   & \\
  $h$ &$h$&   &
\end{tabular}
\end{center}
\end{frame}

\begin{frame}
\frametitle{$3$次交代群}
$g \cdot h = g$とすると、$h = e$になってしまう。\\
$g \cdot h = h$とすると、$g = e$になってしまう。\\
したがって$g \cdot h = e$.同様に$h \cdot g = e$.
\begin{center}
\begin{tabular}{c|ccc}
      &$e$&$g$&$h$\\ \hline
  $e$ &$e$&$g$&$h$\\
  $g$ &$g$&   &$e$\\
  $h$ &$h$&$e$&
\end{tabular}
\end{center}
\end{frame}

\begin{frame}
\frametitle{$3$次交代群}
$g \cdot g = g$とすると、両辺に$g^{-1}$をかけると$g = e$になってしまう。
$g \cdot g = e$とすると、$g \cdot h = e$なので$g = h$になってしまう。\\
したがって$g \cdot g = h$.
同様に$h \cdot h = g$.
\begin{center}
\begin{tabular}{c|ccc}
      &$e$&$g$&$h$\\ \hline
  $e$ &$e$&$g$&$h$\\
  $g$ &$g$&$h$&$e$\\
  $h$ &$h$&$e$&$g$
\end{tabular}
$=$
\begin{tabular}{c|ccc}
                 &$e$           &$\sigma_{123}$&$\sigma_{321}$\\ \hline
  $e$            &$e$           &$\sigma_{123}$&$\sigma_{321}$\\
  $\sigma_{123}$ &$\sigma_{123}$&$\sigma_{321}$&$e$ \\
  $\sigma_{321}$ &$\sigma_{321}$&$e$           &$\sigma_{123}$
\end{tabular}
\end{center}
$3$元からなる群は常に$A_3$と同じ(同型)になる。
\end{frame}

\begin{frame}
\frametitle{作用}
$S_3$は集合$\Omega$上の変換をまとめたものだった。
このようなとき、$S_3$は$\Omega$に\alert{作用する}という。

一般に、群$G (\ni g)$と集合$X (\ni x)$について写像$g(x) \in X$が存在し
\begin{enumerate}
\item $g(h(x)) = (g \cdot h)(x)$
\item $e(x) = x$
\end{enumerate}
が成立するとき、$G$は$X$に作用するという。

歴史的には、作用から群が抽出された(作用 $\rightarrow$ 群)。

現代数学では抽象的に群を導入し、それを色々な集合に作用させる
という使い方が多い(群 $\rightarrow$ 作用)。
\end{frame}

\begin{frame}
\frametitle{自分自身への作用}
$S_3$は並べ方($\eAEB$など)に作用する群だったが、
$S_3$を集合とみなせば、$S_3$は$S_3$自身に作用することができる。

なぜなら、$\sigma_a, \sigma_b \in S_3$について $\sigma_a(\sigma_b) = \sigma \cdot \sigma_b \in S_3$として
\begin{enumerate}
\item $\sigma_a(\sigma_b(\sigma)) = \sigma_a \cdot \sigma_b \cdot \sigma = (\sigma_a \cdot \sigma_b)(\sigma)$
\item $e(\sigma) = e \cdot \sigma = \sigma$
\end{enumerate}
となるため。
\end{frame}

\begin{frame}
\frametitle{整数}
自分自身への作用の例として、整数$\mathbb{Z}$がある。

整数$\mathbb{Z}$は加法$+$について$0$を単位元とする群である。
\begin{enumerate}
\item $(k + l) + m = k + (l + m)$
\item $k + 0 = 0 + k = k$
\item $k + (-k) = (-k) + k = 0$
\end{enumerate}

$3 + 4 = 7$は、
「整数$3$」に「$4$を足す」を作用させると「整数$7$」になる、と考えることもできる。
\end{frame}

\begin{frame}
\frametitle{対称性}
群は対称性を記述する道具であると言われる。

集合$X$上のある全単射 $f: X \rightarrow X; x \mapsto f(x)$について、
\begin{align}
  f(\xi) = \xi
\end{align}
が成立するとき、$\xi$は\alert{$f$不変である}、または\alert{$f$対称である}と言う。
\end{frame}

\begin{frame}
\frametitle{対称性と群}
$\xi \in X$について、$f(\xi) = \xi$となる変換$f$の全体を$F$とすると、\\
$F$は$f(g(\xi)) = (f \cdot g)(\xi)$を積演算として群になる。

\begin{enumerate}
\item 結合側:$((f \cdot g) \cdot h)(\xi) = f(g(h(\xi))) = (f \cdot (g \cdot h))(\xi)$
\item 単位元:恒等変換 $e(\xi) = \xi$ なので $e \in F$
\item 逆元:$f^{-1}(f(\xi)) = f^{-1}(\xi) = \xi$なので$f^{-1} \in F$
\end{enumerate}

このようにして、対称性(不変性)と群が対応する。
\end{frame}


\begin{frame}
\frametitle{Galois理論}
$5$次方程式が解けないことの証明に、なぜ群が必要だったのか。

真面目にやると$1$学期かかるので、「気分」だけを簡単に。
\end{frame}

\begin{frame}
\frametitle{$2$次方程式}
$2$次方程式 $x^2 + bx + c = 0$ を考える。$b, c$は有理数とする。\\
$2$つの解を $\alpha, \beta$ とおくと
\begin{align}
  x^2 + bx + c &= (x - \alpha)(x - \beta) \notag \\
  &= x^2 -(\alpha + \beta)x + \alpha \beta
\end{align}
より、
\begin{align}
  b &= -(\alpha + \beta) \\
  c &= \alpha \beta
\end{align}
を得る。
\end{frame}

\begin{frame}
\frametitle{置換に対する対称性}
$b = -(\alpha + \beta),\ c = \alpha \beta$は、$\alpha \leftrightarrow \beta$の置換について不変。
$\alpha$と$\beta$を入れ替える変換を$\sigma_{\alpha \beta}$と書くと
\begin{align}
  \sigma_{\alpha \beta}(b) &= \sigma_{\alpha \beta} (-(\alpha + \beta)) \notag \\
                           &= -(\beta + \alpha) = b \\
  \sigma_{\alpha \beta}(c) &= \sigma_{\alpha \beta}(\alpha \beta) \notag \\
                           &= \beta \alpha = c
\end{align}
また、$b,\ c$に四則演算を行った$b + c$や $c (b + 3 c/b)$なども、全て$\sigma_{\alpha \beta}$について不変。
\end{frame}

\begin{frame}
\frametitle{置換に対する対称性}
ところで、解$\alpha$そのものに$\sigma_{\alpha \beta}$を作用させると
\begin{align}
  \sigma_{\alpha \beta}(\alpha) = \beta \neq \alpha
\end{align}
すなわち$\alpha$は$\sigma_{\alpha \beta}$について不変ではない。$\beta$も同様。

$b,\ c$に四則演算を行ったものは全て$\sigma_{\alpha \beta}$について不変だったので、
$\alpha$と$\beta$は、$b,\ c$の四則演算だけで表されるものではないことが分かる。
\end{frame}

\begin{frame}
\frametitle{数の空間}
つまり、係数$b,\ c$の四則演算だけでは解$\alpha, \beta$に「到達」できない。\\
(四則演算だけで$2$次方程式の解の公式は得られない)
\[
\begin{tikzpicture}
\fill (-0.3,-0.15) circle (1pt);
\draw (-0.3,-0.15) node[anchor=south]{$b$};
\fill (0.3,-0.25) circle (1pt);
\draw (0.3,-0.25) node[anchor=south]{$c$};
\draw (0,0) circle (1cm and 0.5cm);
\draw (0,-0.5) circle (1.75cm and 1cm);
\draw (0,-0.42) node[anchor=north]{$b$と$c$の四則演算};
\draw (0,-0.9) circle (2cm and 1.4cm);
\fill (-0.7,-2) circle (1pt);
\draw (-0.7,-2) node[anchor=south]{$\alpha$};
\fill (0.7,-2) circle (1pt);
\draw (0.7,-2) node[anchor=south]{$\beta$};
\draw[->] (2.4,0.0) node[anchor=west]{$\sigma_{\alpha \beta}$不変} -- (1.75, -0.5);
\draw[->] (2.4,-0.8) node[anchor=west]{$\sigma_{\alpha \beta}$不変とは限らない} -- (2, -0.9);
\end{tikzpicture}
\]
$\sigma_{\alpha \beta}$不変な集合から「抜け出す」ための演算が\alert{根号$\sqrt{\ }$}である。
\end{frame}

\begin{frame}
\frametitle{対称性を破る}
ところで、$\alpha - \beta$ は$\sigma_{\alpha \beta}$をかけると$-1$倍になる(不変でない)
\begin{align}
  \sigma_{\alpha \beta} \lr{\alpha - \beta} &= \beta - \alpha = -(\alpha - \beta)
\end{align}
しかし、その$2$乗である$(\alpha - \beta)^2$は、$(-1)^2 = 1$なので、
\begin{align}
  \sigma_{\alpha \beta} \lr{\lr{\alpha - \beta}^2} &= \lr{\sigma_{\alpha \beta}(\alpha - \beta)}^2 \notag \\
                                                   &= \clr{-\lr{\alpha - \beta}}^2 \notag \\
                                                   &= \lr{\alpha - \beta}^2
\end{align}
となり、$\sigma_{\alpha \beta}$不変になる。$\Leftrightarrow$ $b,\ c$で書ける。
\end{frame}

\begin{frame}
\frametitle{根号による対称性の破れ}
すなわち、
\begin{align}
  (\alpha - \beta)^2 &= (\alpha + \beta)^2 - 4 \alpha \beta \notag \\
                     &= b^2 - 4 c \\
  \alpha - \beta &= \pm \sqrt{b^2 - 4 c}.
\end{align}
ここの根号$\pm \sqrt{\ }$で\alert{対称性が破れる}。あとは、これと$\alpha + \beta = -b$を
連立すれば、$\alpha,\ \beta$が解ける。

$\rightsquigarrow$ $2$次方程式の解の公式が存在する。
\end{frame}

\begin{frame}
\frametitle{$3$次方程式}
$3$次方程式$x^3+bx^2+cx+d=0$についても、解を$\alpha,\ \beta,\ \gamma$として
\begin{align}
  b &= -(\alpha + \beta + \gamma) \\
  c &= \alpha \beta + \beta \gamma + \gamma \alpha \\
  d &= -\alpha \beta \gamma
\end{align}
であり、$b,\ c,\ d$は $\slr{\alpha \beta \gamma}$の置換$\sigma \in S_3$について不変。
\end{frame}

\begin{frame}
\frametitle{$3$次方程式の解の対称性}
解$\alpha,\ \beta,\ \gamma$について、
$\alpha \rightarrow \beta \rightarrow \gamma \rightarrow \alpha$とする変換$\sigma_{\alpha \beta \gamma}$について
\begin{align}
  \sigma_{\alpha \beta \gamma}(\alpha) = \beta \neq \alpha
\end{align}
などとなり、$\alpha,\ \beta, \gamma$は$S_3$不変ではない。

$2$次方程式のときと同様、係数$b,\ c,\ d$の四則演算で解を記述することはできない。
\end{frame}

\begin{frame}
\frametitle{対称性の分解:$S_3 \rightarrow A_3$}
$2$次方程式における$\alpha - \beta$に対応するものが
\begin{align}
  T = (\alpha - \beta)(\beta - \gamma)(\gamma - \alpha)
\end{align}
である。$T$に互換$\clr{\sigma_{\alpha \beta},\ \sigma_{\beta \gamma},\ \sigma_{\gamma \alpha}}$をかけると$-1$倍になる。e.g.
\begin{align}
  \sigma_{\alpha \beta}(T) &= (\beta - \alpha)(\alpha - \gamma)(\gamma - \beta) = -(\alpha - \beta)(\beta - \gamma)(\gamma - \alpha) = -T \notag
\end{align}
一方、偶置換 $A_3 = \clr{e,\ \sigma_{\alpha \beta \gamma},\ \sigma_{\gamma \beta \alpha}}$については、$T$は不変。e.g.
\begin{align}
  \sigma_{\alpha \beta \gamma}(T) = \sigma_{\alpha \beta}(\sigma_{\beta \gamma}(T)) = (-1)^2 T = T
\end{align}
\end{frame}

\begin{frame}
\frametitle{対称性の分解:$S_3 \rightarrow A_3$}
$T$は奇置換で$-1$倍、偶置換で$1$倍(不変)であったので、
任意の置換$\sigma \in S_3$について、$T^2$は
\begin{align}
  \sigma(T^2) = (\sigma(T))^2 = (\pm T)^2 = T^2
\end{align}
となり、すなわち$S_3$不変。$\Leftrightarrow$ $T^2$は係数$b,\ c,\ d$で書ける。
\end{frame}

\begin{frame}
\frametitle{対称性の分解:$S_3 \rightarrow A_3$}
したがって、$T$は$b,\ c,\ d$と平方根$\sqrt{\ }$で記述できることになる。

しかし、$T$はまだ$A_3 = \clr{e,\ \sigma_{\alpha \beta \gamma},\ \sigma_{\gamma \beta \alpha}}$で不変なので、
解$\alpha,\ \beta,\ \gamma$を$T$で記述することはできない。
\end{frame}

\begin{frame}
\frametitle{対称性の分解:$A_3 \rightarrow e$}
ここで、$1$の立方根$\omega_{\pm} = (-1 \pm \sqrt{-3})/2$を使って
\begin{align}
  U_{\pm} &= \alpha + \beta \omega_{\pm} + \gamma \omega_{\pm}^2
\end{align}
と置く。$U_{\pm}$に$\sigma_{\alpha \beta \gamma}$をかけると$\omega_{\mp}$倍になる。
\begin{align}
  \sigma_{\alpha \beta \gamma}\lr{U_{\pm}} &= \beta + \gamma \omega_{\pm} + \alpha \omega_{\pm}^2 \notag \\
                                     &= \omega_{\pm}^3 \lr{\beta + \gamma \omega_{\pm} + \alpha \omega_{\pm}^2} \notag \\
                                     &= \omega_{\pm}^2 \lr{\alpha + \beta \omega_{\pm} + \gamma \omega_{\pm}^2} = \omega_{\pm}^2 U = \omega_{\mp} U
\end{align}
なお、$\omega_\pm^3 = 1,\ \omega_\pm^2 = \omega_{\mp}$を使った。
\end{frame}

\begin{frame}
\frametitle{対称性の分解:$A_3 \rightarrow e$}
すると、$(\alpha - \beta)^2$や$T^2$と同じ仕組みで、
$U_{\pm}^3$は$\sigma_{\alpha \beta \gamma},\ \sigma_{\gamma \beta \alpha}$で不変になる。
\begin{align}
  \sigma_{\alpha \beta \gamma} \lr{U_{\pm}^3} &= (\omega_{\mp} U_{\pm})^3 = U_{\pm}^3 \\
  \sigma_{\gamma \beta \alpha} \lr{U_{\pm}^3} &= \lr{\sigma_{\alpha \beta \gamma}^2}\lr{U_{\pm}^3} = U_{\pm}^3
\end{align}
$\therefore$ $U_{\pm}^3$は$A_3$不変。$\Leftrightarrow$ $U_{\pm}^3$は$T$と$b, c, d$の四則演算で記述できる。

$\Leftrightarrow$その立方根である$U_{\pm}$は、$T, b, c, d$と立方根$\sqrt[3]{\ }$で記述できる。
\end{frame}

\begin{frame}
\frametitle{対称性の分解:$A_3 \rightarrow e$}
あとは、
\begin{align}
  -b   &= \alpha + \beta + \gamma \\
  U_+  &= \alpha + \beta \omega_+ + \gamma \omega_+^2 \\
  U_-  &= \alpha + \beta \omega_- + \gamma \omega_-^2
\end{align}
だったので、この$3$元連立方程式を解けば$\alpha,\ \beta,\ \gamma$が求められる。\\
$\rightsquigarrow$ 解が係数$b,\ c,\ d$と
$\sqrt{\ },\ \sqrt[3]{\ }$で記述される。\\
$\rightsquigarrow$ $3$次方程式の解の公式が存在する。
\end{frame}





\begin{frame}
\frametitle{対称性の分解}
天下りだが、$1$の立方根$\omega_{\pm} = (-1 \pm \sqrt{-3})/2$を使って
\begin{align}
  U &= \alpha + \beta \omega_+ + \gamma \omega_+^2 \\
  V &= \alpha + \beta \omega_- + \gamma \omega_-^2
\end{align}
と置く。$U$($V$)に$\sigma_{\alpha \beta \gamma}$($\sigma_{\gamma \beta \alpha}$)をかけると$\omega_-$($\omega_+$)倍になる。
\begin{align}
  \sigma_{\alpha \beta \gamma}\lr{U} &= \beta + \gamma \omega_+ + \alpha \omega_+^2 \notag \\
                                     &= \omega_+^3 \lr{\beta + \gamma \omega_+ + \alpha \omega_+^2} \notag \\
                                     &= \omega_+^2 \lr{\beta \omega_+ + \gamma \omega_+^2 + \alpha \omega_+^3} \notag \\
                                     &= \omega_+^2 \lr{\alpha + \beta \omega_+ + \gamma \omega_+^2} = \omega_+^2 U = \omega_- U
\end{align}
最終等号では$\omega_\pm = \omega_{\mp}^2$を使った。
\end{frame}

\begin{frame}
\frametitle{対称性の分解}
すると、$(\alpha - \beta)^2$が対称式になったように、
$U^3,\ V^3$は$\sigma_{\alpha \beta \gamma},\ \sigma_{\gamma \beta \alpha}$で不変になる。
\begin{align}
  \sigma_{\alpha \beta \gamma} \lr{U^3} &= (\omega_- U)^3 = U^3 \\
  \sigma_{\gamma \beta \alpha} \lr{U^3} &= \lr{\sigma_{\alpha \beta \gamma}^2}\lr{U^3} = U^3 \\
  \sigma_{\gamma \beta \alpha} \lr{V^3} &= (\omega_+ V)^3 = V^3 \\
  \sigma_{\alpha \beta \gamma} \lr{V^3} &= \lr{\sigma_{\gamma \beta \alpha}^2}\lr{V^3} = V^3 \\
\end{align}
\end{frame}

\begin{frame}
\frametitle{対称性の分解}
さらに都合の良いことに、$U^3$に$\sigma_{\alpha \beta},\ \sigma_{\beta \gamma}, \sigma_{\gamma \alpha}$をかけると$V^3$になる。
\begin{align}
  \sigma_{\alpha \beta}\lr{U^3}  &= \lr{\beta + \alpha \omega_+ + \gamma \omega_+^2}^3 \notag \\
                                 &= \lr{\beta + \alpha \omega_+ + \gamma \omega_+^2}^3 \omega_+^6 \notag \\
                                 &= \lr{\beta \omega_+^2 + \alpha \omega_+^3 + \gamma \omega_+^4}^3 \notag \\
                                 &= \lr{\alpha + \beta \omega_- + \gamma \omega_-^2}^3 = V^3 \\
  \sigma_{\beta \gamma}\lr{U^3}  &= \lr{\alpha + \gamma \omega_+ + \beta \omega_+^2}^3 \notag \\
                                 &= \lr{\alpha + \beta \omega_- + \gamma \omega_+^2}^3 = V^3
  \sigma_{\gamma \alpha}\lr{U^3} &= \lr{\gamma + \beta \omega_+ + \alpha \omega_+^2}^3 \notag \\
                                 &= \lr{\gamma + \beta \omega_+ + \alpha \omega_+^2}^3 \omega_+^3 \notag \\
                                 &= \lr{\gamma \omega_+ + \beta \omega_+^2 + \alpha \omega_+^3}^3 \notag \\
                                 &= \lr{\alpha + \beta \omega_- + \gamma \omega_-^2}^3 = V^3
\end{align}
したがって、$U^3 + V^3$のような$U^3 \leftrightarrow V^3$の置換について対称な式は
$\sigma_{\alpha \beta},\ \sigma_{\beta \gamma}, \sigma_{\gamma \alpha}$不変。

また、$U^3 - V^3$に$\sigma_{\alpha \beta},\ \sigma_{\beta \gamma}, \sigma_{\gamma \alpha}$をかけると$-1$倍になる。
\end{frame}
\begin{frame}
\frametitle{対称性の分解}
すると、$(\alpha - \beta)^2$が$\sigma_{\alpha \beta}$不変になったように、
$(U^3-V^3)^2$は$\clr{\sigma_{\alpha \beta},\ \sigma_{\beta \gamma}, \sigma_{\gamma \alpha}}$不変になる。

ところで、$U,\ V$は$\clr{\sigma_{\alpha \beta \gamma}, \sigma_{\gamma \beta \alpha}}$不変だったので、
$(U^3 - V^3)^2$は$S_3$不変。
$\rightsquigarrow$ 係数$b,\ c,\ d$で記述できる。
\end{frame}

\begin{frame}
\frametitle{対称性の分解}
$(U^3 - V^3)^2$を係数$b,\ c,\ d$で書いてしまえば、あとは
\end{frame}

\begin{frame}
\frametitle{対称性の分解}
天下りだが、$1$の立方根$\omega = (-1 + \sqrt{-3})/2$を使って
\begin{align}
  U &= \lr{\alpha + \beta \omega + \gamma \omega^2}^3 \\
  V &= \lr{\alpha + \gamma \omega + \beta \omega^2}^3
\end{align}
と置くと$U$と$V$は$\sigma_{\alpha \beta \gamma},\ \sigma_{\gamma \beta \alpha}$について不変。e.g.
\begin{align}
  \sigma_{\alpha \beta \gamma}\lr{U} &= \sigma_{\alpha \beta \gamma}\lr{\lr{\alpha + \beta \omega + \gamma \omega^2}^3} \notag \\
                                     &= \lr{\beta + \gamma \omega + \alpha \omega^2}^3 \notag \\
                                     &= \lr{\beta + \gamma \omega + \alpha \omega^2}^3 \omega^3 \notag \\
                                     &= \lr{\beta \omega + \gamma \omega^2 + \alpha \omega^3}^3 \notag \\
                                     &= \lr{\alpha + \beta \omega + \gamma \omega^2}^3 = U
\end{align}
\end{frame}

\begin{frame}
\frametitle{対称性の分解}
また、$U$と$V$は$\sigma_{\alpha \beta},\ \sigma_{\beta \gamma},\ \sigma_{\gamma \alpha}$で入れ替わる。
\begin{align}
  \sigma_{\alpha \beta}\lr{U} &= \lr{\beta + \alpha \omega + \gamma \omega^2}^3 \notag \\
                              &= \lr{\beta + \alpha \omega + \gamma \omega^2}^3 \omega^6 \notag \\
                              &= \lr{\beta \omega^2+ \alpha \omega^3 + \gamma \omega^4}^3 \notag \\
                              &= \lr{\alpha + \gamma \omega + \beta \omega^2}^3 = V.
\end{align}
したがって$U + V,\ UV$は$\sigma_{\alpha \beta},\ \sigma_{\beta \gamma},\ \sigma_{\gamma \alpha}$についても不変。
つまり、$U + V,\ UV$は$S_3$不変になり、係数$b,\ c,\ d$で記述できる。
\end{frame}

\begin{frame}
\frametitle{対称性の分解}
四則演算の関数 $f_+,\ f_\ast$で
\begin{align}
  U + V &= f_+(b, c, d) \\
  U V &= f_\ast(b, c, d)
\end{align}
と書けているので、この$2$元連立$2$次方程式を解けば$U,\ V$が$b,\ c,\ d$の四則演算と$\sqrt{\ }$で表される。

この$2$次方程式と解く際に$\sqrt{\ }$をとるので、$U \leftrightarrow V$の対称性、
すなわち$\sigma_{\alpha \beta},\ \sigma_{\beta \gamma},\ \sigma_{\gamma \alpha}$の不変性が
壊れる。($U$と$V$の「区別がつかない」空間から、「区別がつく」空間に拡がる)
\end{frame}

\begin{frame}
\frametitle{対称性の分解}
$U,\ V$が解けると、
\begin{align}
  -b &= \alpha + \beta + \gamma \\
  \sqrt[3]{U} &= \alpha + \beta \omega + \gamma \omega^2 \\
  \sqrt[3]{V} &= \alpha + \beta \omega^2 + \gamma \omega
\end{align}
なので、この$3$元連立方程式を解けば$\alpha,\ \beta,\ \gamma$が求められる。
ここの$\sqrt[3]{\ }$をとるときに$\sigma_{\alpha \beta \gamma},\ \sigma_{\gamma \beta \alpha}$の
不変性が壊れる。
\end{frame}

\begin{frame}
\frametitle{数の空間の拡大}
% $\clr{b,c,d} \rightarrow \clr{U+V, UV} \in $\{$b,c,d$の四則演算\}$ \rightarrow \clr{U, V} \in \clr{\sigma_{\alpha \beta \gamma}, \sigma{\gamma \beta \alpha} inv.} \rightarrow \clr{\sqrt[3]{U}, \alpha, \beta, \gamma}$
\[
\begin{tikzpicture}
\fill (-0.3,-0.15) circle (1pt);
\draw (-0.3,-0.15) node[anchor=south]{$b$};
\fill (0.3,-0.25) circle (1pt);
\draw (0.3,-0.25) node[anchor=south]{$c$};
\draw (0,0) circle (1cm and 0.5cm);
\draw (0,-0.5) circle (1.75cm and 1cm);
\draw (0,-0.42) node[anchor=north]{$b$と$c$の四則演算};
\draw (0,-0.9) circle (2cm and 1.4cm);
\fill (-0.7,-2) circle (1pt);
\draw (-0.7,-2) node[anchor=south]{$\alpha$};
\fill (0.7,-2) circle (1pt);
\draw (0.7,-2) node[anchor=south]{$\beta$};
\draw[->] (2.4,0.0) node[anchor=west]{$\sigma_{\alpha \beta}$不変} -- (1.75, -0.5);
\draw[->] (2.4,-0.8) node[anchor=west]{$\sigma_{\alpha \beta}$不変とは限らない} -- (2, -0.9);
\end{tikzpicture}
\]
\end{frame}

\begin{frame}
\frametitle{方程式の可解性}
$3$次方程式が解けるのは、$U + V$と$UV$という$2$つの$S_3$不変な式(対称式)が見付かったからである。

$4$次方程式についても、$4$次対称群$S_4$について $3$つの$S_4$不変な式が見付かり、
これを$3$次方程式を使って解くことで、$4$次方程式が解ける。

しかし、$5$次方程式では、$5$次対象群$S_5$にそのような``都合のいい対称式''は存在せず、
そのために$5$次方程式の解の公式は存在しない。

この``都合のいい対称式''が存在しないことの証明に、群論が使われる。
\end{frame}

\begin{frame}
Mは「2質点であること」を制限する条件。

\end{frame}

\begin{frame}{連成振動子系}

\FPeval\u{\A*cos(\OMEGA*\t)+\B*sin(1.73*\OMEGA*\t)}%
\FPeval\v{\A*cos(\OMEGA*\t)-\B*sin(1.73*\OMEGA*\t)}%
\centering
\begin{tikzpicture}[>=stealth]
%%% 左壁
\coordinate (south east of left wall) at (0,-.5*\wallHeight);
\coordinate (north west of left wall) at ($(south east of left wall) + (-\wallWidth,\wallHeight)$);
\fill[wall] (south east of left wall) rectangle (north west of left wall);
\draw[thick] (south east of left wall) -- (south east of left wall |- north west of left wall);
%%% 右壁
\coordinate (south west of right wall) at (\totalLength,-.5*\wallHeight);
\coordinate (north east of right wall) at ($(south west of right wall) + (\wallWidth,\wallHeight)$);
\fill[wall] (south west of right wall) rectangle (north east of right wall);
\draw[thick] (south west of right wall) -- (south west of right wall |- north east of right wall);
%%% おもり
\node[ball] (a) at (\totalLength/3 + \u,0) {};
\node[ball] (b) at (2*\totalLength/3 + \v,0) {};
%%% 座標軸
\draw[->] (0.5,\axisDepth) -- +(\totalLength-1,0);
\draw[dotted,thick] (\totalLength/3,\axisDepth-0.3) -- +(0,1.6)
              (2*\totalLength/3,\axisDepth-0.3) -- +(0,1.6)
              (a.south |- south east of left wall) -- (a.south)
              (b.south |- south east of left wall) -- (b.south);
\draw[->] (\totalLength/3,\axisDepth-0.15) --node[below] {$x_1$} +(\u,0);
\draw[->] (2*\totalLength/3,\axisDepth-0.15) --node[below] {$x_2$} +(\v,0);
%%% 座標計算
\coordinate (0) at (0,0);
\coordinate (0-right) at (\springStraightLength,0);
\coordinate (a-left) at ($(a.west) + (0.1-\springStraightLength,0)$);
\coordinate (a-right) at ($(a.east) + (\springStraightLength,0)$);
\coordinate (b-left) at ($(b.west) + (0.1-\springStraightLength,0)$);
\coordinate (b-right) at ($(b.east) + (\springStraightLength,0)$);
\coordinate (c-left) at (\totalLength + 0.1-\springStraightLength,0);
\coordinate (c) at (\totalLength,0);
%%% バネの直線部
\draw (0) -- (0-right)
      (a-left) -- (a.west)
      (a.east) -- (a-right)
      (b-left) -- (b.west)
      (b.east) -- (b-right)
      (c-left) -- (c);
%%% バネのグルグル部
\draw[spring={\totalLength/3 + \u}] (0-right) -- node[springk]{} (a-left);
\draw[spring={\totalLength/3 + \v - \u}] (a-right) -- node[springk]{} (b-left);
\draw[spring={\totalLength/3 - \v}] (b-right) -- node[springk]{} (c-left);
\end{tikzpicture}
\end{frame}

\begin{frame}
\frametitle{運動方程式}
\begin{align}
  \label{eq:motionN2}
  m 
\end{align}
\end{frame}

\begin{frame}
\frametitle{参考文献}
\printbibliography
\end{frame}

\end{document}
%%% Local Variables:
%%% mode: latex
%%% TeX-master: t
%%% End:
