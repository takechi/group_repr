\documentclass[12pt, t]{beamer}
\usepackage{luatexja}
\usepackage{amsmath}
\usepackage{bm}
\usepackage{setspace}

\setstretch{1.2}

\usetheme{CambridgeUS}

\usepackage{geometry}
\usepackage{tikz}
\usetikzlibrary{calc, decorations.pathmorphing, patterns}
\usepackage[nomessages]{fp}

%%%%%%% 調和振動子系
%%%% \A*cos(\OMEGA*\t)+\B*sin(1.73*\OMEGA*\t) のパラメータ
\def\A{1}
\def\B{0.8}
\def\OMEGA{0.2}

%%% 壁などの描画のパラメータ
\def\wallHeight{2}
\def\wallWidth{0.2}
\def\totalLength{10}
\def\springStraightLength{0.2}
\def\axisDepth{-0.7}
\def\t{30}

%%% 壁などのスタイル
\tikzset{wall/.style={pattern = north east lines}}
\tikzset{ball/.style={circle,shade,outer color=black!90!white,inner color=white,inner sep=2.5mm,label={$m$}}}
\tikzset{spring/.style={decorate,decoration={aspect=0.4, segment length=#1, amplitude=2mm,coil}}}
\tikzset{springk/.style={label={$k$},yshift=2}}

%%%%%%%


\title{群の表現}
\author{武智}
\begin{document}
\frame{\maketitle}

\begin{frame}{流れ}
\begin{enumerate}
\item 群とは
\item 群の表現とは
\end{enumerate}
\end{frame}

\begin{frame}
\frametitle{はじめに}
現代数学では、ほとんど至るところで群論が使われている。

物理学でも使われているが、群論の言葉を明示的には使わずに議論を進めることも多い(当社比)。

情報科学では、暗号理論など一部を除いて、あからさまに群論が使われることは少ない(個人の感想です)。

\end{frame}

\begin{frame}
\frametitle{はじめに}
Q. 場の理論、楕円暗号のような、高度に抽象化された場面でしか群論は使えないのだろうか?

\vspace{1\zw}
$\rightsquigarrow$ A. そんなことはない。

ただ、あまりにもシンプルな系では、群論を意識せずに問題が解けてしまうことが多いため、
その御利益に気付きにくい。
\end{frame}

\begin{frame}{連成振動子系}

\FPeval\u{\A*cos(\OMEGA*\t)+\B*sin(1.73*\OMEGA*\t)}%
\FPeval\v{\A*cos(\OMEGA*\t)-\B*sin(1.73*\OMEGA*\t)}%
\centering
\begin{tikzpicture}[>=stealth]
%%% 左壁
\coordinate (south east of left wall) at (0,-.5*\wallHeight);
\coordinate (north west of left wall) at ($(south east of left wall) + (-\wallWidth,\wallHeight)$);
\fill[wall] (south east of left wall) rectangle (north west of left wall);
\draw[thick] (south east of left wall) -- (south east of left wall |- north west of left wall);
%%% 右壁
\coordinate (south west of right wall) at (\totalLength,-.5*\wallHeight);
\coordinate (north east of right wall) at ($(south west of right wall) + (\wallWidth,\wallHeight)$);
\fill[wall] (south west of right wall) rectangle (north east of right wall);
\draw[thick] (south west of right wall) -- (south west of right wall |- north east of right wall);
%%% おもり
\node[ball] (a) at (\totalLength/3 + \u,0) {};
\node[ball] (b) at (2*\totalLength/3 + \v,0) {};
%%% 座標軸
\draw[->] (0.5,\axisDepth) -- +(\totalLength-1,0);
\draw[dotted,thick] (\totalLength/3,\axisDepth-0.3) -- +(0,1.6)
              (2*\totalLength/3,\axisDepth-0.3) -- +(0,1.6)
              (a.south |- south east of left wall) -- (a.south)
              (b.south |- south east of left wall) -- (b.south);
\draw[->] (\totalLength/3,\axisDepth-0.15) --node[below] {$x_1$} +(\u,0);
\draw[->] (2*\totalLength/3,\axisDepth-0.15) --node[below] {$x_2$} +(\v,0);
%%% 座標計算
\coordinate (0) at (0,0);
\coordinate (0-right) at (\springStraightLength,0);
\coordinate (a-left) at ($(a.west) + (0.1-\springStraightLength,0)$);
\coordinate (a-right) at ($(a.east) + (\springStraightLength,0)$);
\coordinate (b-left) at ($(b.west) + (0.1-\springStraightLength,0)$);
\coordinate (b-right) at ($(b.east) + (\springStraightLength,0)$);
\coordinate (c-left) at (\totalLength + 0.1-\springStraightLength,0);
\coordinate (c) at (\totalLength,0);
%%% バネの直線部
\draw (0) -- (0-right)
      (a-left) -- (a.west)
      (a.east) -- (a-right)
      (b-left) -- (b.west)
      (b.east) -- (b-right)
      (c-left) -- (c);
%%% バネのグルグル部
\draw[spring={\totalLength/3 + \u}] (0-right) -- node[springk]{} (a-left);
\draw[spring={\totalLength/3 + \v - \u}] (a-right) -- node[springk]{} (b-left);
\draw[spring={\totalLength/3 - \v}] (b-right) -- node[springk]{} (c-left);
\end{tikzpicture}
\end{frame}

\begin{frame}
\frametitle{運動方程式}
\begin{align}
  \label{eq:motionN2}
  m 
\end{align}
\end{frame}

\begin{frame}
\frametitle{群論の歴史}
嚆矢はGalois.

代数方程式の解の間の対称性
\end{frame}

\end{document}